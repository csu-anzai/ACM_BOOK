%!TEX program = xelatex
\documentclass[]{article}
\usepackage[UTF8]{ctex}
\usepackage{fontspec}
\usepackage{listings}
\usepackage{geometry}
\geometry{left=1.0cm,right=1.0cm,top=1.0cm,bottom=2.0cm}
\setmainfont{Avenir}

\begin{document}

\section{线性基}
性质:
\begin{enumerate}
    \item 原序列里面的任意一个数都可以由线性基里面的一些数异或得到。
    \item 线性基里面的任意一些数异或起来都不能得到0 
    \item 线性基里面的数的个数唯一,并且在保持性质一的前提下,数的个数是最少的
\end{enumerate}
用法:

\begin{enumerate}
    \item 线性基可以套在线段树上使用,线段树区间合并的时候将一个区间的所有值插入到另外一个区间即可。
    \item 不支持删除操作。遇到需要删除操作的线性基考虑用时间分治。
\end{enumerate}

\begin{lstlisting}[language={c}]
    struct Linear_Basis {
    LL b[63],nb[63],tot;
    void init() {
        tot=0;
        memset(b,0,sizeof(b));
        memset(nb,0,sizeof(nb));
    }
    bool ins(LL x) {
        for(int i=62;i>=0;i--)
            if (x&(1LL<<i)){
                if (!b[i]) {b[i]=x;break;}
                x^=b[i];
            }
        return x>0;
    }
    LL Max(LL x){
        LL res=x;
        for(int i=62;i>=0;i--)
            res=max(res,res^b[i]);
        return res;
    }
    LL Min(LL x){
        LL res=x;
        for(int i=0;i<=62;i++)
            if (b[i]) res^=b[i];
        return res;
    }
     void rebuild(){
        for(int i=62;i>=0;i--)
            for(int j=i-1;j>=0;j--)
                if (b[i]&(1LL<<j)) b[i]^=b[j];
        for(int i=0;i<=62;i++)
            if (b[i]) nb[tot++]=b[i];
    }
    LL Kth_Max(LL k){
        LL res=0;
        for(int i=62;i>=0;i--)
            if (k&(1LL<<i)) res^=nb[i];
        return res;
    }
} LB;
\end{lstlisting}


\section{线段树}
用法:
\begin{enumerate}
    \item 区间查询最值、和 等等
    \item 权值线段树。 一般配合离散化使用(离散化注意区间处理为左闭右开)
    \item 可持久化线段树可以处理版本问题。
    \item 可以遍历线段树处理一些问题(类似于分治的思想)。
\end{enumerate}

\subsection{带LAZY的朴素线段树}

\begin{lstlisting}[language={c}]
    //线段树 区间加 区间求和
    //带lazy
    #include <iostream>
    #include <cstring>
    using namespace std;
    #define MAXN 100100
    #define lc o * 2
    #define rc o * 2 + 1
    #define mid (l + r) / 2
    #define LL long long
    struct Segment_Tree {
        LL tag[MAXN * 4 + 10];
        LL sum[MAXN * 4 + 10];
        Segment_Tree() {
            memset(tag, 0, sizeof(tag));
            memset(sum, 0, sizeof(sum));
        }
        void update(int o, int l, int r, LL del) {
            sum[o] += (r - l + 1) * del;
            tag[o] += del;
        }
        void push_down(int o, int l, int r) {
            update(lc, l, mid, tag[o]);
            update(rc, mid + 1, r, tag[o]);
            tag[o] = 0;
        }
        void insert(int o, int l, int r, int qx, int qy, LL del) {
            if (qx <= l && r <= qy) {
                sum[o] += (r - l + 1) * del;
                tag[o] += del;
                return;
            }
            push_down(o, l, r);
            if (qx <= mid)
                insert(lc, l, mid, qx, qy, del);
            if (qy > mid)
                insert(rc, mid + 1, r, qx, qy, del);
            sum[o] = sum[lc] + sum[rc];
        }
        LL query(int o, int l, int r, int qx, int qy) {
            if (qx <= l && r <= qy)
                return sum[o];
            push_down(o, l, r);
            LL t = 0;
            if (qx <= mid)
                t += query(lc, l, mid, qx, qy);
            if (qy > mid)
                t += query(rc, mid + 1, r, qx, qy);
            return t;
        }
    } ST;
\end{lstlisting}



\subsection{可持久化线段树(不带修)}

\begin{enumerate}
    \item 注意内存开MAXN << 5
    \item 核心思想是对1...n做n个线段树,每两个线段树只有height个节点不同
\end{enumerate}

\begin{lstlisting}[language={c}]
    //可持久化线段树求第k大
    #include <algorithm>
    #include <cstring>
    #include <iostream>
    using namespace std;
    #define MAXN 200010
    #define mid (l + r) / 2
    #define LL long long
    int n, m;
    int asize;
    int a[MAXN];
    int b[MAXN];

    int get_id(int num) { return lower_bound(a + 1, a + asize + 1, num) - a; }

    int root[MAXN];

    struct SegmentTree {

        struct node {
            int lc, rc;
            int sum;
            node() { sum = 0; }
        };

        node ST[MAXN << 5];
        int cnt = 0;

        int build(int l, int r) {
            int now = ++cnt;
            if (l == r)
                return now;
            ST[cnt].lc = build(l, mid);
            ST[cnt].rc = build(mid + 1, r);
            return now;
        }

        void init() { build(1, asize); }

        int insert(int o, int l, int r, int k) {
            int now = ++cnt;
            ST[now] = ST[o];
            ST[now].sum++;
            if(l == r) 
                return now;
            if (k <= mid)
                ST[now].lc = insert(ST[o].lc, l, mid, k);
            else
                ST[now].rc = insert(ST[o].rc, mid + 1, r, k);
            return now;
        }

        int query(int u, int v, int l, int r, int k) {
            int num = ST[ST[v].lc].sum - ST[ST[u].lc].sum;
            if (l == r)
                return l;
            if (k <= num)
                return query(ST[u].lc, ST[v].lc, l, mid, k);
            else
                return query(ST[u].rc, ST[v].rc, mid + 1, r, k - num);
        }
    } ST;

    int main() {
        cin >> n >> m;
        for (int i = 1; i <= n; i++){
            scanf("%d", &a[i]);
            b[i] = a[i];
        }
        sort(a + 1, a + n + 1);
        asize = unique(a + 1, a + n + 1) - a - 1;
        ST.init();
        for (int i = 1; i <= n; i++) {
            root[i] = ST.insert(root[i - 1], 1, asize, get_id(b[i]));
        }
        for (int i = 1; i <= m; i++) {
            int l, r, k;
            scanf("%d%d%d", &l, &r, &k);
            int ans = ST.query(root[l - 1], root[r], 1, asize, k);
            printf("%d\n", a[ans]);
        }
    }
\end{lstlisting}

\end{document}
