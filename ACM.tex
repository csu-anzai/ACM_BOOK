%!TEX program = xelatex
\documentclass[]{article}
\usepackage[UTF8]{ctex}
\usepackage{fontspec}
\usepackage{listings}
\usepackage{geometry}
\geometry{left=1.0cm,right=1.0cm,top=1.0cm,bottom=2.0cm}
\setmainfont{Avenir}

\begin{document}

\tableofcontents

\section{计算几何}
\label{sec:geometry}

\subsection{几何基础}
\label{sec:geometry_basic}
\subsubsection{点积}
\label{sec:dianji}
点积的应用  

\begin{enumerate}
\item 判断两个向量是否垂直$a\bot b <=> a·b = 0$
\item 求两个向量的夹角,点积$<0$为钝角,点积$>0$为锐角
\end{enumerate}

\begin{lstlisting}[language={c}]
    //点积
    double dot(vector a,vector b){
        return a.x*b.x+a.y*b.y;
    }
    //求夹角
    double Angle(vector a,vector b){
        return acos(dot(a,b)/len(a)/len(b));
    }
    //求模长
    double len(vector a){ 
        return sqrt(dot(a,a));
    }
\end{lstlisting}

\subsubsection{叉积}
\label{sec:chaji}

\begin{enumerate}
    \item 判断平行$a\times b = 0$
    \item 判断左右$a\times b > 0$ 在左边,$< 0$在右边
\end{enumerate}

\subsubsection{点和直线}

\subparagraph{}
直线上所有的点表示为$P = P_0+tv$。若已知直线的两个点A、B,则方程为$A+(B-A)t$
\subparagraph{}
点与直线的位置关系判断$Ax_0+By_0+C$的正负。

\begin{enumerate}
    \item 点到直线的距离
    \item 点到线段的距离
    \item 判断线段相交
    \item 求两直线交点
    \item 点在与线段位置
\end{enumerate}

\begin{lstlisting}[language={c}]
    //取符号
    int dcmp(double d){
        if(fabs(d) > eps)
            return 0;
        if(d > 0)
            return 1;
        return -1;
    }
    //点到直线的距离
    利用叉积求面积,然后除以平行四边形的底边长,得到平行四边形的高即点到直线的距离
    double distl(point p,point a,point b)
    {
    	vector v=p-a; vector u=b-a;
    	return fabs(cross(v,u))/len(u);
    }

    //点到线段的距离
    比点到直线的距离稍微复杂。
    因为是线段,所以如果平行四边形的高在区域之外的话就不合理
    这时候需要计算点到距离较近的端点的距离
    double dists(point p,point a,point b)
    {
	    if (a==b) return len(p-a);
	    vector v1=b-a,v2=p-a,v3=p-b;
	    if (dcmp(dot(v1,v2))<0) return len(v2);
	    else if (dcmp(dot(v1,v3))>0) return len(v3);
	    return fabs(cross(v1,v2))/len(v1);
    }

    //判断两个线段是否相交
    跨立实验:判断一条线段的两端是否在另一条线段的两侧
    两个端点与另一线段的叉积乘积为负,需要正反判断两侧。
    bool segment(point a,point b,point c,point d)
    {
	    double c1=cross(b-a,c-a),c2=cross(b-a,d-a);
	    double d1=cross(d-c,a-c),d2=cross(d-c,b-c);
	    return dcmp(c1)*dcmp(c2)<0&&dcmp(d1)*dcmp(d2)<0;
    }

    //求两条直线的交点
    point line_intersection(point a,point a0,point b,point b0)  
    {  
        double a1,b1,c1,a2,b2,c2;  
        a1 = a.y - a0.y;  
        b1 = a0.x - a.x;  
        c1 = cross(a,a0);  
        a2 = b.y - b0.y;  
        b2 = b0.x - b.x;  
        c2 = cross(b,b0);  
        double d = a1 * b2 - a2 * b1;  
        return point((b1 * c2 - b2 * c1) / d,(c1 * a2 - c2 * a1) / d);  
    }

    //点与线段位置 不含端点判断
    bool on_segment(Point p,Point a,Point b){ 
        return dcmp(cross(a-p,b-p))==0&&dcmp(dot(a-p,b-p)<0); 
    }
\end{lstlisting}

\subsubsection{多边形}
\begin{enumerate}
    \item 三角形
    \item 求多边形面积
    \item 判断点与多边形位置关系
    \item 
\end{enumerate}

\subparagraph{}
海伦公式:
$$
S = \sqrt{p(p - a)(p - b)(p - c)}, p = \frac{(a + b + c)}{2}
$$
\subparagraph{}
欧拉定理:
$$
V+F-E=2  \\  
V:vertex  \\  
F:face  \\  
E:edge  \\  
$$
\begin{lstlisting}[language={c}]
    //求多边形的面积
    再多边形内取一个点进行三角剖分,用叉积求三角形的面积。
    因为叉积是有向面积,所以任意多边形都使用。注意最后取绝对值。
    double PolygonArea(Point* p,int n)
    {
        double area=0;
        for(int i=1;i<n-1;i++)
            area+=Cross(p[i]-p[0],p[i+1]-p[0]);
        return area/2;
    }

    //判断点在多边形内部
    射线法:以该点为起点引一条射线,与多边形的边界相交奇数次,说明在多边形的内部。
    int pointin(point p,point* a,int n)   
    {  
        int wn=0,k,d1,d2;  
        for (int i=1;i<=n;i++)  
        {  
            if (dcmp(dists(p,a[i],a[(i+1-1)%n+1]))==0)
                return -1;//判断点是否在多边形的边界上   
            k=dcmp(cross(a[(i+1-1)%n+1]-a[i],p-a[i]));  
            d1=dcmp(a[i].y-p.y);  
            d2=dcmp(a[(i+1-1)%n+1].y-p.y);  
            if (k>0&&d1<=0&&d2>0)  wn++;  
            if (k<0&&d2<=0&&d1>0)  wn--;  
        }  
        if (wn)  return 1;  
        else return 0;  
    }
\end{lstlisting}

\subsection{Pick定理}
\subparagraph{}
Pick 定理:
\subparagraph{}
给定顶点座标均是整点(或正方形格子点)的简单多边形
\subparagraph{}
皮克定理说明了其面积$A$和内部格点数目$i$,边上格点数目$b$的关系。
$$
A=i+\frac{b}{2}-1
$$


\subsection{凸包}


\end{document}